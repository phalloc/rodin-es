%%%%%%%%%%%%%%%%%%%%%%%%%%%%%%%%%%%%%%%%%
% Short Sectioned Assignment
% LaTeX Template
% Version 1.0 (5/5/12)
%
% This template has been downloaded from:
% http://www.LaTeXTemplates.com
%
% Original author:
% Frits Wenneker (http://www.howtotex.com)
%
% License:
% CC BY-NC-SA 3.0 (http://creativecommons.org/licenses/by-nc-sa/3.0/)
%
%%%%%%%%%%%%%%%%%%%%%%%%%%%%%%%%%%%%%%%%%

%----------------------------------------------------------------------------------------
%	PACKAGES AND OTHER DOCUMENT CONFIGURATIONS
%----------------------------------------------------------------------------------------

\documentclass[paper=a4, fontsize=11pt]{scrartcl} % A4 paper and 11pt font size

\usepackage[margin=0.6in]{geometry}
\usepackage[utf8]{inputenc}
\usepackage[T1]{fontenc} % Use 8-bit encoding that has 256 glyphs
\usepackage{fourier} % Use the Adobe Utopia font for the document - comment this line to return to the LaTeX default
\usepackage[english]{babel} % English language/hyphenation
\usepackage{amsmath,amsfonts,amsthm} % Math packages

\usepackage{lipsum} % Used for inserting dummy 'Lorem ipsum' text into the template
\usepackage{multicol}
\usepackage{sectsty} % Allows customizing section commands
\allsectionsfont{\centering \normalfont\scshape} % Make all sections centered, the default font and small caps

\usepackage{fancyhdr} % Custom headers and footers
\pagestyle{fancyplain} % Makes all pages in the document conform to the custom headers and footers
\fancyhead{} % No page header - if you want one, create it in the same way as the footers below
\fancyfoot[L]{} % Empty left footer
\fancyfoot[C]{} % Empty center footer
\fancyfoot[R]{\thepage} % Page numbering for right footer
\renewcommand{\headrulewidth}{0pt} % Remove header underlines
\renewcommand{\footrulewidth}{0pt} % Remove footer underlines
\setlength{\headheight}{13.6pt} % Customize the height of the header
\usepackage{enumitem}
\setlist{leftmargin=35mm}

\numberwithin{equation}{section} % Number equations within sections (i.e. 1.1, 1.2, 2.1, 2.2 instead of 1, 2, 3, 4)
\numberwithin{figure}{section} % Number figures within sections (i.e. 1.1, 1.2, 2.1, 2.2 instead of 1, 2, 3, 4)
\numberwithin{table}{section} % Number tables within sections (i.e. 1.1, 1.2, 2.1, 2.2 instead of 1, 2, 3, 4)

\setlength\parindent{0pt} % Removes all indentation from paragraphs - comment this line for an assignment with lots of text

%----------------------------------------------------------------------------------------
%	TITLE SECTION
%----------------------------------------------------------------------------------------

\newcommand{\horrule}[1]{\rule{\linewidth}{#1}} % Create horizontal rule command with 1 argument of height

\title{\vspace{-1.5cm}
\normalfont \normalsize 
\textsc{Instituto Superior Técnico\\Universidade de Lisboa} \\ [12pt] % Your university, school and/or department name(s)
\huge Specification of Software 2015/16\\1\textsuperscript{st} Project Report \\ [5pt]
}

\author{
  Daiane Oliveira\\
  \texttt{ist423160}
  \and
  Tiago Diogo\\
  \texttt{ist173559}
}

\date{\normalsize\today} % Today's date or a custom date

\begin{document}

\maketitle % Print the title

\section{Project Approach}
Given the operations and restrictions presented in the project statement the group decided to use a hierarchical approach, where each machine would be responsible for a specific group of operations and restrictions. This allows the use of a refinement process where each machine considers a single dimension and is then refined and expanded with the operations and restrictions of the next dimension. The used machines are explained in the following section, ordered by the refining sequence (each item refines the previous).

\section{Developed Machines and Context}
\begin{itemize}
	\item[\textbf{Machine mac\_users}]The abstract model. In this machine we addressed the management of the users. The static part (SETS and AXIOMS defining the use of the CONSTANTS for the finite sets) was developed in Context ctx1. The state was modelled with 3 variables representing the USERS and their properties. The machine is responsible for the 4 user management operations and respects the restrictions [1-9]
	\item[\textbf{Machine mac\_files}]The mac\_users refinement. In this machine we addressed the file management section. The state was modelled with 7 variables representing the FILES, their properties as well as the local versions and archive capabilities. The machine is responsible for the 4 file management operations and respects the restrictions [10-19,38]
	\item[\textbf{Machine mac\_shares}]The mac\_files refinement. In this machine we addressed the sharing functionality. The state was modelled with 2 variable representing the sharing MODE and the USERS with access. This machine is responsible for the 3 file sharing operations and respects the restrictions [20-37]
	\item[\textbf{Machine mac\_backups}]The mac\_shares refinement and final machine (GitBob). In this machine we addressed the backup and restore capabilities. The state was modelled with 2 variables that kept record of the files being backed up and their log history. The machine is responsible for the 3 backup and restore operations and respects the restrictions[39-50]
\end{itemize}

\section{Interactive Prover}
The following profs were not automatically proved and required the group intervention
\begin{multicols}{2}
\begin{itemize}
	\item[\textbf{Machine mac\_files}] uploadFile/inv4/INV
	\item[\textbf{Machine mac\_shares}] addFile/inv2/INV
	\item[\textbf{Machine mac\_shares}] downgradeBasic/inv6/INV
	\item[\textbf{Machine mac\_shares}] uploadFile/inv5/INV
	\item[\textbf{Machine mac\_backups}] uploadFileWithBackup/inv2/INV
	\item[\textbf{Machine mac\_backups}] addFile/inv3/INV
	\item[\textbf{Machine mac\_backups}] uploadFileWithBackup/inv2/INV
	\item[]
\end{itemize}
\end{multicols}

\end{document}